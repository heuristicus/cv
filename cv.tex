% !!!!Compile with XeTeX!!!!
% layout from http://aleplasmati.comuv.com/doc/cv.pdf
\documentclass[a4paper,10pt]{article}

%A Few Useful Packages
%\usepackage{marvosym}
\usepackage{fontspec} 					%for loading fonts
\usepackage{xunicode,xltxtra,url,parskip} 	%other packages for formatting
\usepackage{amsfonts}
\usepackage{multirow}
\RequirePackage{color,graphicx}
\usepackage[usenames,dvipsnames]{xcolor}
\usepackage{moreverb}
\usepackage[boldfont]{xeCJK}
\usepackage{ruby}
%\usepackage{tipa}
%\usepackage[big]{layaureo} 				%better formatting of the A4 page
% an alternative to Layaureo can be ** \usepackage{fullpage} **
\usepackage{supertabular} 				%for Grades
\usepackage{titlesec}					%custom \section
\usepackage{fullpage}
%Setup hyperref package, and colours for links
\usepackage{hyperref}
\usepackage{array}
\definecolor{linkcolour}{rgb}{0,0.2,0.6}
\hypersetup{colorlinks,breaklinks,urlcolor=linkcolour, linkcolor=linkcolour}

%FONTS
\defaultfontfeatures{Mapping=tex-text}
%\setCJKmainfont{Meiryo}
\setmainfont[Numbers=Lining,Ligatures=Common]{Adobe Garamond Pro}
\setmonofont{Inconsolata}
\linespread{1.2}
\headheight -0.5in
\textheight 10.3in
\newcolumntype{x}[1]{%
>{\raggedleft\hspace{0pt}}p{#1}}%

%CV Sections inspired by: 
%http://stefano.italians.nl/archives/26
\titleformat{\section}{\Large\scshape\raggedright}{}{0em}{}[\titlerule]
\titlespacing{\section}{0pt}{3pt}{3pt}
\titleformat{\subsection}{\large\scshape\raggedright}{}{0em}{}[\titlerule]
\titlespacing{\subsection}{0pt}{3pt}{3pt}
\titleformat{\subsubsection}{\centering\large\scshape}{}{0em}{}
\titlespacing{\subsection}{0pt}{3pt}{3pt}
%Tweak a bit the top margin
%\addtolength{\voffset}{-1.3cm}

\usepackage[absolute]{textpos}

\setlength{\TPHorizModule}{30mm}
\setlength{\TPVertModule}{\TPHorizModule}
\textblockorigin{2mm}{0.65\paperheight}
\setlength{\parindent}{0pt}

\begin{document}
\thispagestyle{empty}
\pagestyle{empty}
%\par{\centering {\Huge \ruby{Micha\l}{\begin{IPA}[mixaw]\end{IPA}} \ruby{\textsc{Staniaszek}}{\begin{IPA}[sta\:na\:sek]\end{IPA}} }\bigskip\par}
\par{\centering {\Huge Michal Staniaszek}\par}

\section{Personal Data}

\begin{tabular}{x{2.8cm}l}
  \textsc{Date of Birth:} & 7 December 1990\\
%  \textsc{Home Address:}   & 36 Selby Road, London, W5 1LX\\
  \textsc{Address:} & 77 Harborne Road, Birmingham, B68 9JF\\
  \textsc{Telephone:}     & +44 7745 916895\\
  \textsc{e-mail:}     & \href{mailto:m.staniaszek@gmail.com}{m.staniaszek@gmail.com}\\
  \textsc{web:}       & \url{www.github.com/heuristicus}
\end{tabular}

\section{Education and Qualifications} 
\begin{tabular}{x{2.8cm}|p{12cm}}
  \textsc{September 2009} to& \textbf{BSc Computer Science with Study Abroad}\\
  \textsc{June 2013}&Year 1 -- First Class (77\%)\\
  &Year 2 -- First Class (80\%)\\
  &Final Year -- First Class (expected)\\
  &\textsc{University of Birmingham}, Birmingham, UK\\\multicolumn{2}{c}{}\\[-0.2cm]
  \textsc{September 2011} to& \textbf{Japanese Language Programme}\\ 
  \textsc{July 2012}&\textbf{Japanese Language Proficiency Test} level N1 (90\%)\\
  &\textsc{Keio University}, Tokyo, Japan\\\multicolumn{2}{c}{}\\[-0.2cm]
  \textsc{September 2007} to& A-level \textbf{Mathematics, Physics, Computing and Polish}\\ 
  \textsc{July 2009}& \textsc{St. Dominic's Sixth Form College}, London, UK\\
  %\textsc{July} 2007& \textbf{Cardinal Wiseman Roman Catholic High School}\\
  %\textsc{July} 2002& \textbf{St. Gregory's Roman Catholic Primary School}\\
%\hyperlink{bham_ug}{\hfill \footnotesize List of Results p.3}
\end{tabular}

\section{Academic and Research Activity}
\begin{tabular}{x{2.8cm}|p{12cm}}
  \textsc{September 2012} to& Final Year Project---\textbf{Time Delay Estimation in Gravitationally Lensed Photon Streams}\\
  \textsc{April 2013}&\footnotesize{Developed a system to calculate time delay between photon streams by estimating the underlying function using weighted least squares techniques. Worked with simulated and real-world data. Supervised by Peter Tiňo.}\\\multicolumn{2}{c}{}\\[-0.2cm]
  \textsc{January} to&Participant at \textbf{SAUC-E 2011} (Student Autonomous Underwater Competition - Europe)\\
  \textsc{July 2011}&\footnotesize{Collaboration between the University of Birmingham and Southampton University to prepare the Delphin AUV for participation in the competition in July. Converted the control system from Matlab to ROS Python.}\\\multicolumn{2}{c}{}\\[-0.2cm]
  \textsc{October 2010} to&Founding member of \textbf{Birmingham Autonomous Robotics Club}\\
  \textsc{July 2013}&\footnotesize{Co-founded the club due to interest from computer science faculty and undergraduate students in solving robotics problems and increasing the profile of the field in the school.}\\
\end{tabular}

\section{Teaching}
\begin{tabular}{x{2.8cm}|p{12cm}}
  \textsc{January 2013} to& \textbf{Robot Programming Demonstrator} at \\
  \textsc{April 2013}&\textsc{Department of Computer Science}, University of Birmingham\\
  &\footnotesize{Answered questions about the implementation of various algorithms and the application of robotics techniques for use on LEGO NXT robots running LeJOS.}\\\multicolumn{2}{c}{}\\[-0.2cm]
  \textsc{September 2012} to& \textbf{Foundation Year Demonstrator} at \\
  \textsc{December 2012}&\textsc{Department of Computer Science}, University of Birmingham\\
  &\footnotesize{Helped foundation year students understand basic programming concepts, and evaluated their performance in assignments.}\\\multicolumn{2}{c}{}\\[-0.2cm]
  \textsc{September 2010} to& \textbf{First Year Workshop Demonstrator} at \\
  \textsc{April 2011}&\textsc{Department of Computer Science}, University of Birmingham\\
  &\footnotesize{One of five second year students selected to demonstrate for the First Year Workshop Java module. Provided guidance to weaker students and answered questions about programming concepts and the Java API.}\\
\end{tabular}

\section{Bursaries and Scholarships}
\begin{tabular}{x{2.8cm}|p{12cm}}
  \textsc{January} to& \textbf{Japan Student Services Organisation Scholarship}\\
  \textsc{April 2012}&\footnotesize{Short-term scholarship to support study in Japan.}\\\multicolumn{2}{c}{} \\[-0.2cm]
  \textsc{July} to& \textbf{Nuffield Undergraduate Research Bursary}\\
  \textsc{September 2011}&\footnotesize{Bursary to support the development of a particle filter based localisation algorithm for an AUV.}
\end{tabular}

\section{Work Experience}
\begin{tabular}{x{2.8cm}|p{12cm}}
  \textsc{December 2011} to& \textbf{Hall staff} at \textsc{Gyu-Kaku}, Tokyo, Japan\\
  \textsc{June 2012}
  &\footnotesize{Serving food and taking orders, working entirely in Japanese with Japanese co-workers and clients.}\\\multicolumn{2}{c}{} \\[-0.2cm]
  \textsc{July} to& \textbf{Technical Assistant} at \textsc{Japan Services} Estate Agents, London, UK\\
  \textsc{September 2010}&\footnotesize{Assisted the company director with the development of a website. Used image manipulation tools to create images for the site. Solved hardware and software problems encountered by co-workers.}\\
\end{tabular}

\section{Interests and Activities}
Artificial intelligence, cycling, history of science, Japanese language and culture, languages, linux, music, photography, physics, robot morality and ethics, science education.

\begin{minipage}[t]{0.48\textwidth}
  \section{Programming and Software}
  \begin{tabular}{lp{0.7\textwidth}}
    Basics:& Bash, C++, Common Lisp, CUDA, Emacs, Git, Haskell, Linux, Subversion\\
    Confident:& C, Java, Python, ROS, \LaTeX
  \end{tabular}
\end{minipage}
\hskip 0.04\textwidth
\begin{minipage}[t]{0.48\textwidth}
  \section{Languages}
  \begin{tabular}{rl}
    English:&Native\\
    German:&Beginner\\
    Japanese:&Fluent\\
    Polish:&Fluent\\
    % \textsc{French:}&Basic Knowledge\\
  \end{tabular}
\end{minipage}

\section{Referees}
\begin{center}
  \begin{tabular}{l|l}
  Dr Jeremy Wyatt$^1$ & Dr Nick Hawes$^1$\\ % HACKING!
  \textsc{email}: j.l.wyatt@cs.bham.ac.uk&\textsc{email}: n.a.hawes@cs.bham.ac.uk\\
  \textsc{tel}: +44 (0)121 414 4788&\textsc{tel}: +44 (0) 121 414 3739\\
  \textsc{fax} +44 (0)121 414 4281&\textsc{fax} +44 (0)121 414 4281\\
\end{tabular}\\\vspace{0.2cm}
\footnotesize{$^1$School of Computer Science, University of Birmingham, Edgbaston, Birmingham, B15 2TT, UK}\\
\end{center}

\begin{comment}
\newpage
\par{\centering\Large \hypertarget{bham_ug}{Computer Science with
    Study Abroad}\par}\large{\centering Grades\par}\normalsize
\begin{center}
  \section{First Year}
  \begin{tabular}{lcc}
    \multicolumn{1}{c}{\textsc{Exam}}&\textsc{Grade}&\textsc{Credits}\\ \hline
    Design and Media Team&78&	10\\
    Introduction to Software Engineering	&48&	10\\
    Software Workshop 1 (Java)	&83&	30\\
    Language and Logic	&95&	10\\
    Robot Programming	&94&	10\\
    Foundations of Computer Science	&55	&20\\
    Introduction to AI	&74&	10\\
    Japanese Language	&90&	20\\\cline{2-3}
    & Total&120\\
    &\textsc{Average}&\textbf{77.33}
  \end{tabular}
\end{center}
\begin{center}
  \section{Second Year}
  \begin{tabular}{lcc}
    \multicolumn{1}{c}{\textsc{Exam}}&\textsc{Grade}&\textsc{Credits}\\ \hline
    Software Workshop Team Java&80&	10\\
    Communication Skills and Professional Issues	&76&	10\\
    Software System Components 1	&83&	20\\
    Software System Components 2	&79&	20\\
    Computer Systems and Architecture	&70&	10\\
    Mathematical Techniques for Computer Science	&74	&10\\
    Japanese Language	&83&	40\\\cline{2-3}
    & Total&120\\
    &\textsc{Average}&\textbf{80.00}
  \end{tabular}
\end{center}
\bigskip \hrule \bigskip
\par{\centering\Large \hypertarget{doms}{St. Dominic's Sixth Form
    College}\par}\large{\centering Grades\par}\normalsize
\begin{center}
  \subsubsection{A-Level}
  \begin{tabular}{lc}
    \multicolumn{1}{c}{\textsc{Exam}}&\textsc{Grade}\\ \hline
    Mathematics & B\\
    Physics	&B\\
    Computing&B\\
    Polish& A
  \end{tabular}\\
\end{center}
\begin{center}
  \subsubsection{AS Level}
  \begin{tabular}{lc}
    \multicolumn{1}{c}{\textsc{Exam}}&\textsc{Grade}\\ \hline
    Chemistry& C\\
  \end{tabular}\\
\end{center}\newpage
\bigskip \hrule \bigskip
\par{\centering\Large \hypertarget{wise}{Cardinal Wiseman Roman
    Catholic High School}\par}\large{\centering Grades\par}\normalsize
\begin{center}
  \subsubsection{GCSE}
  \begin{tabular}{ll}
    \multicolumn{1}{c}{\textsc{Exam}}&\textsc{Grade}\\ \hline
    Mathematics&A\\
    Statistics&A*\\
    Science (Double Award)& A/A\\
    English Literature&A*\\
    English Language& A\\
    French& A*\\
    Religious Studies& A\\
    History&A
  \end{tabular}
\end{center}
\end{comment}
\end{document}