% !!!!Compile with XeTeX!!!!
% layout from http://aleplasmati.comuv.com/doc/cv.pdf
\documentclass[a4paper,10pt]{article}

%A Few Useful Packages
%\usepackage{marvosym}

\usepackage{multirow}
\RequirePackage{color,graphicx}
\usepackage[usenames,dvipsnames]{xcolor}
\usepackage{moreverb}
% an alternative to Layaureo can be ** \usepackage{fullpage} **
%\usepackage{supertabular} 				%for Grades
\usepackage{titlesec}					%custom \section
\usepackage[left=2cm, right=2cm,bottom=1cm,top=1.2cm]{geometry}
\setlength{\parskip}{0.3cm plus4mm minus3mm}
%\setlength{\parskip}{0.2cm}
%Setup hyperref package, and colours for links
\usepackage{hyperref}
\usepackage{array}
\definecolor{linkcolour}{rgb}{0,0,0}
\hypersetup{colorlinks,breaklinks,urlcolor=linkcolour, linkcolor=linkcolour}

%FONTS
\usepackage[T1]{fontenc} 					%for loading fonts
\usepackage{libertine}
% \defaultfontfeatures{Mapping=tex-text}
%\setCJKmainfont{Meiryo}
% \setmainfont[Numbers=Lining,Ligatures=Common]{Adobe Garamond Pro}
% \setmonofont{Adobe Garamond Pro}
\linespread{1.2}
\newcolumntype{x}[1]{%
>{\raggedleft\hspace{0pt}}p{#1}}%

%CV Sections inspired by: 
%http://stefano.italians.nl/archives/26
\titleformat{\section}{\Large\scshape\raggedright}{}{0em}{}[\titlerule]
\titlespacing{\section}{0pt}{3pt}{3pt}

\usepackage[absolute]{textpos}

\setlength{\TPHorizModule}{30mm}
\setlength{\TPVertModule}{\TPHorizModule}
\textblockorigin{2mm}{0.65\paperheight}
\setlength{\parindent}{0pt}

\begin{document}
\thispagestyle{empty}
\pagestyle{empty}
% \par{\centering {\Huge \ruby{Micha\l}{\begin{IPA}[mixaw]\end{IPA}} \ruby{\textsc{Staniaszek}}{\begin{IPA}[sta\:na\:sek]\end{IPA}} }\bigskip\par}
% \begin{tabular}{rc}
%   {\Huge Michal Staniaszek}& Born 7 December 1990, British national
% \href{mailto:contact@michalstaniaszek.com}{m.staniaszek@gmail.com}&  
% \end{tabular}
  {\Huge \textsc{Michal Staniaszek}} \hspace*{0.3cm} Born 7 December 1990, British national \hspace*{0.3cm} \href{mailto:contact@michalstaniaszek.com}{m.staniaszek@gmail.com}  


\vspace*{1pt}
\section{Employment History}
\begin{tabular}{x{2.8cm}|p{13cm}}
  \textsc{June 2017} to& \textbf{Research Assistant} at \textsc{Technische Universit{\"a}t Wien}, Vienna, Austria\\
  \textsc{January 2019}&\footnotesize{PhD candidate and teaching assistant in the Vision for Robotics group, working on semantic segmentation and mapping for long-term autonomous robot operation.}\\\multicolumn{2}{c}{} \\[-0.2cm]
  \textsc{October 2016} to& \textbf{STRANDS Project Research Associate} at \textsc{University of Birmingham}, Birmingham, UK\\
  \textsc{May 2017}&\footnotesize{Assisted project partners with system integration. Facilitated robot deployment on site and developed software tools for running and monitoring the system. Aggregated and improved software package documentation.}\\\multicolumn{2}{c}{} \\[-0.2cm]
  \begin{comment}
    General work on the STRANDS project codebase, improving the user-facing functionality. Mostly ROS development in Python and C++, along with bash scripting and some linux networking and system administration activities.

    Involved in the deployment of one of the STRANDS project robots in an office environment, where I helped monitor and diagnose issues with the robot and tried to ensure that the deployment achieved its research goals. I worked with researchers from the other universities involved in the project to help them get their parts of the system tested and working during the deployment.
  \end{comment}
  \textsc{March} to& \textbf{Marie Skłodowska-Curie Early Stage Researcher} at \textsc{Fraunhofer IPA}, Germany\\
  \textsc{July 2016}&\footnotesize{PhD candidate in environment modelling and motion prediction in dynamic environments as part of the Horizon 2020 SECURE project.}\\\multicolumn{2}{c}{} \\[-0.2cm]
  \begin{comment}
    Working as a fellow of the SECURE project, part of the European Union’s Horizon 2020 research and innovation programme. In brief, the project aims to improve the safety of robots with a view to using them in environments which require interaction with humans. My responsibility was multimodal modelling and motion prediction in dynamic environments. 

    In brief, I was to use computer vision and other sensing modalities available to the robot to map dynamic environments, storing information about the nature of the dynamics, and then feed this information to higher and lower level information processing so that it could be used to improve the safety of the overall system. Some of the project partners were using robots developed at Fraunhofer (the care-o-bot series), and so part of my job was to provide software support for the robots.

    I left the position after 4 months due to an environment that I felt was not conducive to me completing a PhD.
  \end{comment}
  \textsc{July 2015} to& \textbf{Innovation Team Intern} at \textsc{Yujin Robot Co., Ltd}, Seoul, Republic of Korea\\
  \begin{comment}
    Part of the team working on development of software for the GoCart delivery robot, designed to operate in nursing homes and other healthcare and hospitality environments. Worked on behaviour trees and behaviour implementation, diagnostics and module interfacing in C++ and Python with ROS. Made some small contributions to the audio_common and diagnostics packages in ROS.
  \end{comment}
  \textsc{January 2016}&\footnotesize{Python and C++ development in ROS for the GoCart logistics robot. Worked on behaviour trees and behaviour design, diagnostics, simulation layer implementation and module interfacing. Made contributions to several ROS packages.}\\
  % \multirow{2}{*}{\textsc{July 2013}} & \textbf{Systems and Software Assistant} at \textsc{Kaon Ltd}, Guildford, UK\\
  %                                     &\footnotesize{Developed a GUI prototype for a portable device in Java, to demonstrate functionality to the customer. Set up and tested filesystems on linux for optimal operation on a large RAID device.}\\\multicolumn{2}{c}{} \\[-0.2cm]
  % \textsc{December 2011} to& \textbf{Waiter} at \textsc{Gyu-Kaku}, Shibaura, Tokyo, Japan\\
  % \textsc{June 2012}&\footnotesize{Served food and took orders, working entirely in Japanese with Japanese co-workers and clientele.}\\\multicolumn{2}{c}{} \\[-0.2cm]
  % \textsc{July} to& \textbf{Technical Assistant} at \textsc{Japan Services Rent Ltd}, London, UK\\
  % \textsc{September 2010}&\footnotesize{Assisted the company director with the development of a website. Used image manipulation tools to create images for the site. Solved hardware and software problems encountered by co-workers.}\\
\end{tabular}

\section{Education and Qualifications} 
\begin{tabular}{x{2.8cm}|p{13cm}}
  \textsc{August 2013} to& \textbf{MSc Systems, Control and Robotics}, Robotics and Autonomous Systems track\\
  \textsc{June 2015}& \textsc{KTH Royal Institute of Technology}, Stockholm, Sweden\\\multicolumn{2}{c}{}\\[-0.2cm]
  \textsc{September 2009} to& \textbf{BSc Computer Science with Study Abroad}, First Class Honours\\
  \textsc{June 2013}&Top of graduating class (final year: 92\%, weighted average: 89\%)\\
                         &\textsc{University of Birmingham}, Birmingham, UK\\\multicolumn{2}{c}{}\\[-0.2cm]
  \textsc{September 2011} to& \textbf{Japanese Language Programme}, Advanced Level\\ 
  \textsc{July 2012}& Japanese Language Proficiency Test level N1 (Highest level)\\
                         &\textsc{Keio University}, Tokyo, Japan
  %\textsc{September 2007} to& A-level \textbf{Mathematics, Physics, Computing and Polish}\\ 
  %\textsc{July 2009}& \textsc{St. Dominic's Sixth Form College}, London, UK\\
  %\textsc{July} 2007& \textbf{Cardinal Wiseman Roman Catholic High School}\\
  %\textsc{July} 2002& \textbf{St. Gregory's Roman Catholic Primary School}\\
\end{tabular}

\section{Academic and Research Activity}
\begin{tabular}{x{2.8cm}|p{13cm}}
  \textsc{January} to& Master's Thesis: \textbf{Feature-Feature Matching for Object Retrieval in Point Clouds}\\
  \textsc{June 2015}&\footnotesize{Developed a system using PCL and ROS to extract features from point clouds, and find query objects in the resulting feature set. Performed an experimental evaluation to find good descriptors and interest point methods. Supervised by John Folkesson.}\\\multicolumn{2}{c}{}\\[-0.2cm]
  \textsc{July} to& Participant in \textbf{Tohoku University Engineering Summer Programme (TESP)}\\
  \textsc{August 2014}&\footnotesize{Attended lectures on several different areas related to robotics, and worked on a short project on obstacle avoidance for a tracked mobile robot.}\\\multicolumn{2}{c}{}\\[-0.2cm]
  \textsc{September 2012} to& Bachelor's Thesis: \textbf{Time Delay Estimation in Gravitationally Lensed Photon Streams}\\
  \textsc{April 2013}&\footnotesize{Developed a system to estimate characteristic functions of streams with weighted least squares techniques and compare them with probabilistic techniques to estimate the time delay. Supervised by Peter Tiňo.}\\\multicolumn{2}{c}{}\\[-0.2cm]
  \textsc{January} to&Participant at \textbf{Student Autonomous Underwater Competition - Europe (SAUC-E)} \\
  \textsc{July 2011}&\footnotesize{Collaboration between the University of Birmingham and University of Southampton to prepare the Delphin AUV for the competition in July. Converted the control system from Matlab to ROS Python.}\\\multicolumn{2}{c}{}\\[-0.2cm]
  \textsc{October 2010} to&Founding member of \textbf{Birmingham Autonomous Robotics Club}\\
  \textsc{July 2013}&\footnotesize{Co-founded the club due to interest from computer science faculty and undergraduate students in solving robotics problems and increasing the profile of the field in the school.}\\
\end{tabular}

\section{Teaching}
\begin{tabular}{x{2.8cm}|p{13cm}}
  \textsc{June 2017} to& \textbf{Teaching Assistant} at \textsc{Vision for Robotics Group, Technische Universit{\"a}t Wien}\\
  \textsc{January 2019}&\footnotesize{Administering the \emph{Machine Vision and Cognitive Robotics} and \emph{Robotics and Computer Vision} courses. Updating and maintaining assignment content, managing tutors, grading assignments, and providing assistance to students.}\\\multicolumn{2}{c}{}\\[-0.2cm]
  \textsc{January 2017} to& \textbf{Robot Programming Demonstrator} at \textsc{University of Birmingham}\\
  \textsc{April 2017}&\footnotesize{Discussed theoretical and software problems with students to help them complete assignments and understand the AI concepts being taught. Graded student assignments.}\\\multicolumn{2}{c}{}\\[-0.2cm]
  \textsc{January 2013} to& \textbf{Robot Programming Demonstrator} at \textsc{University of Birmingham}\\
  \textsc{April 2013}&\footnotesize{Provided assistance to students with the implementation of various algorithms and the application of artificial intelligence techniques for use on LEGO NXT robots.}\\\multicolumn{2}{c}{}\\[-0.2cm]
  \textsc{September 2012} to& \textbf{Foundation Year Computer Science Demonstrator} at \textsc{University of Birmingham}\\
  \textsc{December 2012}&\footnotesize{Helped foundation year students understand basic programming concepts, and evaluated their performance in assignments.}\\\multicolumn{2}{c}{}\\[-0.2cm]
  \textsc{September 2010} to& \textbf{Software Workshop (Java) Demonstrator} at \textsc{University of Birmingham}\\
  \textsc{April 2011}&\footnotesize{One of five second year students selected to demonstrate for the First Year Workshop Java module. Provided advice on assignments, programming concepts, and the Java API.}\\
\end{tabular}

\section{Bursaries, Scholarships and Awards}
\begin{tabular}{x{2.8cm}|p{13cm}}
  \multirow{2}{*}{\textsc{July 2013}} & \textbf{Computer Science Prize \& Research Committee Project Prize}\\
                                      & \footnotesize{Awarded by the University of Birmingham School of Computer Science to the highest scoring final year student, and for the best research-related final dissertation.}\\\multicolumn{2}{c}{} \\[-0.2cm]
  \textsc{January} to& \textbf{Japan Student Services Organisation Scholarship}\\
  \textsc{April 2012}&\footnotesize{Short-term scholarship to support study in Japan.}\\\multicolumn{2}{c}{} \\[-0.2cm]
  \textsc{July} to& \textbf{Nuffield Undergraduate Research Bursary}\\
  \textsc{September 2011}&\footnotesize{Bursary to support a summer research project to investigate particle filter localisation for AUVs.}
\end{tabular}

\begin{minipage}[t]{0.47\textwidth}
  \section{Programming and Software}
  Python, C++, ROS, git, Linux, \LaTeX
  \vskip 0.2cm
  \centering
  Public repositories at \href{http://www.github.com/heuristicus}{github.com/heuristicus}
\end{minipage}
\hskip 0.04\textwidth
\begin{minipage}[t]{0.47\textwidth}
  \section{Languages}
    English (native), Japanese (fluent), Polish (bilingual)
\end{minipage}

% \section{Interests}
% Cycling, Japanese language and culture, languages, music, photography, reading,
% bouldering, politics, history

\section{Referees}
Addresses and positions for referees are current. Bold text indicates
where the person supervised me.
\begin{center}
  \begin{tabular}{lll}
    \textbf{University of Birmingham} & \textbf{Technische Universit{\"a}t Wien} & \textbf{Yujin Robot}\\
    Nick Hawes& Markus Vincze & Daniel Stonier \\
    Associate Professor of Robotics & Associate Professor & Program Manager --- Simulation\\
    nickh@robots.ox.ac.uk & vincze@acin.tuwien.ac.at & d.stonier@gmail.com\\
    Oxford Robotics Institute& Automation and Control Institute & Toyota Research Institute  \rule{0pt}{0.5cm} \\
    23 Banbury Road, Felstead House& Technische Universit{\"a}t Wien & One Kendall Square  \\
    Oxford & Gu{\ss}hausstrasse 27-29 & Building 100, Suite 1-201 \\
    OX2 6NN, UK& 1040, Vienna, Austria & Cambridge, MA 02139, USA
  \end{tabular}
  
    % \begin{tabular}{lll}
    %   John Folkesson
    %   Assistant Professor
    %   johnf@kth.se
    %   Centre for Autonomous Systems
    %   Kungliga Tekniska H\"{o}gskolan
    %   100 44
    %   Stockholm, Sweden
    % \end{tabular}

\end{center}


\begin{comment}
\newpage
\par{\centering\Large \hypertarget{bham_ug}{Computer Science with
    Study Abroad}\par}\large{\centering Grades\par}\normalsize
\begin{center}
  \section{First Year}
  \begin{tabular}{lcc}
    \multicolumn{1}{c}{\textsc{Module}}&\textsc{Grade}&\textsc{Credits}\\ \hline
    Design and Media Team&78&	10\\
    Foundations of Computer Science	&55	&20\\
    Introduction to AI	&74&	10\\
    Introduction to Software Engineering	&48&	10\\
    Japanese Language	&90&	20\\
    Language and Logic	&95&	10\\
    Robot Programming	&94&	10\\
    Software Workshop Java	&83&	30\\\cline{2-3}
    & Total&120\\
    &\textsc{Average}&\textbf{77}
  \end{tabular}
\end{center}
\begin{center}
  \section{Second Year}
  \begin{tabular}{lcc}
    \multicolumn{1}{c}{\textsc{Module}}&\textsc{Grade}&\textsc{Credits}\\ \hline
    Communication Skills and Professional Issues	&76&	10\\
    Computer Systems and Architecture	&70&	10\\
    Japanese Language	&83&	40\\
    Mathematical Techniques for Computer Science	&74	&10\\
    Software System Components 1	&83&	20\\
    Software System Components 2	&79&	20\\
    Software Workshop Team Java&80&	10\\\cline{2-3}
    & Total&120\\
    &\textsc{Average}&\textbf{80}
  \end{tabular}
\end{center}
\begin{center}
  \section{Final Year}
  \begin{tabular}{lcc}
    \multicolumn{1}{c}{\textsc{Module}}&\textsc{Grade}&\textsc{Credits}\\ \hline
    Intelligent Data Analysis	&90&	10\\
    Intelligent Robotics&85&	20\\
    Introduction to Neural Computation&91&	10\\
    Operating Systems with C/C++&92	&20\\
    Parallel Programming&93&	10\\
    Planning	&97&	10\\
    Software Project&94&	40\\\cline{2-3}
    & Total&120\\
    &\textsc{Average}&\textbf{92}
  \end{tabular}
\end{center}
\bigskip \hrule \bigskip
\par{\centering\Large \hypertarget{doms}{St. Dominic's Sixth Form
    College}\par}\large{\centering Grades\par}\normalsize
\begin{center}
  \subsubsection{A-Level}
  \begin{tabular}{lc}
    \multicolumn{1}{c}{\textsc{Module}}&\textsc{Grade}\\ \hline
    Mathematics & B\\
    Physics	&B\\
    Computing&B\\
    Polish& A
  \end{tabular}\\
\end{center}
\begin{center}
  \subsubsection{AS Level}
  \begin{tabular}{lc}
    \multicolumn{1}{c}{\textsc{Module}}&\textsc{Grade}\\ \hline
    Chemistry& C\\
  \end{tabular}\\
\end{center}\newpage
\bigskip \hrule \bigskip
\par{\centering\Large \hypertarget{wise}{Cardinal Wiseman Roman
    Catholic High School}\par}\large{\centering Grades\par}\normalsize
\begin{center}
  \subsubsection{GCSE}
  \begin{tabular}{ll}
    \multicolumn{1}{c}{\textsc{Module}}&\textsc{Grade}\\ \hline
    Mathematics&A\\
    Statistics&A*\\
    Science (Double Award)& A/A\\
    English Literature&A*\\
    English Language& A\\
    French& A*\\
    Religious Studies& A\\
    History&A
  \end{tabular}
\end{center}
\end{comment}
\end{document}
%%% Local Variables:
%%% mode: latex
%%% TeX-master: t
%%% TeX-engine: xetex
%%% End:
